\chapter{Introduction}
This is the introduction. 
\newpage
\section{2番目}
This is the second page of the introductory chapter. 
\newpage
\section{Third page}
This is the third page of the introductory chapter.

You can use \texttt{$\backslash$putFigure} command to put a figure into the document. 
The formula is following: 
\begin{equation}
\backslash\texttt{putFigure}[\texttt{option}]\{\texttt{file location}\}
\{\texttt{caption}\}\{\texttt{label name}\}
\end{equation}
where in the \texttt{option} we can set the width of the figure with the unit of \texttt{$\backslash$textwidth}. 
The example shows in \figurename~\ref{fig:recotime}. 
\putFigure[.8]{reco_time.pdf}{PDF format example}{fig:recotime}
The \texttt{JPEG} format can also be handled (see Fig.~\ref{fig:pork}). 
\putFigure[.5]{pork.jpeg}{JPEG format example}{fig:pork}

The command \texttt{$\backslash$putTable} is also available. 
But it is less useful than the \texttt{$\backslash$putFigure} 
since there are a lot of parameters for definition a table.
Here, an example is shown. 
If you define a table with 2 columns and 2 rows, the command you should 
type is following: 
\begin{equation}
\backslash\texttt{putTable}\{\texttt{cc}\}
\{\texttt{1 \& 2}\backslash\backslash\ \backslash\texttt{midrule 3 \& 4}\}
\{\texttt{Example of the table}\}\{\texttt{tab:exmpl}\}
\end{equation}
then we get \tablename~\ref{tab:exmpl}. 
\putTable{cc}{1&2\\ \midrule 3&4}{Example of the table}{tab:exmpl}


