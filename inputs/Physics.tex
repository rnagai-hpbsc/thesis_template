\chapter{Physics motivation}

\section{Unit (\texttt{siunitx} package)}

Units are important for the physics. This template includes \texttt{siunitx} package so that you can easily write units properly. You can find several descriptions for the way to use \texttt{siunitx} on the internet (for example, \url{http://www.yamamo10.jp/yamamoto/comp/latex/make_doc/unit/index.php} in Japanese).  

\section{Physics package}

This template already includes the \texttt{physics} package. A helpful reference can be \url{https://qiita.com/HelloRusk/items/ce9f49e9b3fc0344ae23}. 


\section{Overleaf support}

This template should work on the Overleaf platform. You should upload all of the files in the template and make sure the ``Main document'' in the Menu is \texttt{main.tex}. If you write it in Japanese, the ``Compiler'' in the menu should be set to LaTeX (the default is pdfLaTeX, but it is not working if the document contains non-alphabet characters). 
